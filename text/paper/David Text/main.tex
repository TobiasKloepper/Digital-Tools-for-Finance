\documentclass[fleqn,10pt]{olplainarticle}

\usepackage{threeparttable}
\usepackage{blindtext}
\usepackage{verbatim}
\usepackage[english]{babel}
\usepackage{subfig}
\usepackage{multirow}
\usepackage{hyperref}
\usepackage{authblk}

\begin{document}

\flushbottom
\begin{titlepage}

    \begin{center}
        \vspace*{1cm}
        {\Large Digital tools for finance}\\[2.5cm]
        \Huge
        How to write a scientific paper in Latex (and more)
        
        \vspace{0.5cm}
        
        \Large

        
        \vspace{0.5cm}
        
        \normalsize

        \vspace{1.5cm}
        
        {\large Igor Pozdeev} \\[3cm]
        
        \begin{tabular}{ll}
        \hline
        {\small Authors (Affiliation):} & {\small Tobias Klöpper (University of Zurich)}\\
        & {\small Michele Senki (University of Zurich)}\\
        & {\small Marco Antonio Barcellos Junior (University of Zurich)}\\
        & {\small David Annoni (University of Zurich)}\\
        \hline
        \end{tabular}
        
        \vfill
        
        
        \vspace{0.8cm}
        
        
        Faculty of economics\\
        University of Zurich\\
        \today
        
    \end{center}
\end{titlepage}

\section*{Abstract}
\blindtext[2]

\section*{Thanks}
We are really grateful for this opportunity to write a scientific paper that will be nominated for the Nobel-prize. Thanks to the previous work of \cite{Auctions}, \cite{Corona} and \cite{Tailwagseconomy}, we were able to solve the global problem of... (well read the article and find out)

\newpage
\section{Introduction}
Welcome to this test article written by some very enthusiastic authors. Although in the introduction there should never be a list, these will be the topics discussed in this article.
\blindlist{enumerate}[5]

\section{Methods and Materials}

We used several different methods which will be explained later in this paper.

\subsection{Maths explained}
We used the following maths in our paper.\\
\blindmathtrue
\blindtext[2]\\
If you wish to see a more detailed description of maths used in another paper, that is in no way related to this abstract, please see \cite{Tailwagseconomy} and \cite{Corona}. If you are not satisfied, also check out \cite{Auctions}.

\section{Main text}
\blindtext[15]\\
If you want to deepen the understanding of this text, please see the figure \ref{fig:figures}. 

\newpage
\section{Figures to undermine the fact that the text above has no meaning at all}
\subsection{Figures}
\begin{figure}[h]%
    \centering
    \subfloat[Visual representation of the text above]{{\includegraphics[width=5cm]{145160} }}%
    \qquad
    \subfloat[A joke that caught me off-guard]{{\includegraphics[width=5cm]{joke} }}%
    \caption{Some... figures?}
    \label{fig:figures}
\end{figure}

\section{Fancy tables}

\begin{table}[h]
\centering
  \begin{threeparttable}
    \caption{Sample ANOVA table}
    \label{tab:table numero uno}
     \begin{tabular}{lllll}
        \toprule
        Important name here & \( df \) & \( f \) & \( \eta \) & \( p \) \\
        \midrule
                 &     \multicolumn{4}{c}{Some random numbers}     \\
        Row 1    & 1        & 0.67    & 0.55       & 0.41    \\
        Row 2    & 2        & 0.02    & 0.01       & 0.39    \\
        Row 3    & 3        & 0.15    & 0.33       & 0.34    \\
        Row 4    & 4        & 1.00    & 0.76       & 0.54    \\
        \bottomrule
     \end{tabular}
    \begin{tablenotes}
      \small
      \item I found this table on \href {https://tex.stackexchange.com/questions/12676/add-notes-under-the-table}{Stackexchange(link)}
      and thought, why make a table myself when others have already done it for me?
    \end{tablenotes}
  \end{threeparttable}
\end{table}
Now, i can really recommend to take a closer look at table \ref{tab:table numero uno}, because it contains some life-changing information.


\section{Because we love it, here's some maths}
\subsection{As announced... some (unnecessary) maths}
\begin{equation}
  x = a_0 + \cfrac{1}{a_1 
          + \cfrac{1}{a_2 
          + \cfrac{1}{a_3 + \cfrac{1}{a_4} } } }
\end{equation}

\newpage
\section*{Acknowledgments}

A big thank you to all the people that ever supported and never doubted us - you are the real heroes. This paper might not be the best out there, but definitely Nobel-prize worthy. We will not forget you all when we visit Sweden and claim our very well deserved prize.

\newpage
\bibliography{sample}

\end{document}