\documentclass[11pt, oneside]{article}   	% use "amsart" instead of "article" for AMSLaTeX format
\usepackage{geometry}                		% See geometry.pdf to learn the layout options. There are lots.
\geometry{letterpaper}                   		% ... or a4paper or a5paper or ... 
%\geometry{landscape}                		% Activate for rotated page geometry
\usepackage[parfill]{parskip}    			% Activate to begin paragraphs with an empty line rather than an indent
\usepackage{graphicx}				% Use pdf, png, jpg, or eps§ with pdflatex; use eps in DVI mode
								% TeX will automatically convert eps --> pdf in pdflatex		
\usepackage{amssymb}
\usepackage{mathtools}
\usepackage{enumerate}
\usepackage{tikz}

\usetikzlibrary{arrows}

\def\firstcircle{(90:1.75cm) circle (2.5cm)}
\def\secondcircle{(210:1.75cm) circle (2.5cm)}
\def\thirdcircle{(330:1.75cm) circle (2.5cm)}

%SetFonts

%SetFonts


\title{Financial Crises: Past, Present, Future - Problem Set 3}
\author{Michele Senkal - \texttt{michelemelek.senkal@uzch.ch}}
\date{November 2, 2021}							% Activate to display a given date or no date

\begin{document}

\maketitle

\section*{1.1)}
Tangible capital is by definition the sum of physical assets, such as inventories, buildings, production equipment. Whereas, intangible capital is composed by not physical in nature assets rich of added value (e.g., patents, R&D investment, monopoly rights, goodwill). Hence, intangible capital also contributes to the fundamental value, yet not easily observable/measured

	

\section*{1.2)}
As opposed to fundamental value, which is the intrinsic value of the underlying productive assets (tangible and intangible) of the corporate sector. The market value described by the authors, is the market capitalization of U.S. corporations at the end of August 1929. The former is obtained by multiplying the P/E ratio of a known sample of corporations (Sloan (1936) is the main research used) by the total NIPA earnings for the year.The writers of the article chose to use a market value to GNP ratio of 1.67, in order to hold conservatism toward Fisher’s view about stock market not being overvalued, rather than the more accurate 1.54 (Sloan research) as the paper tries to prove Fisher’s theory that the market was undervalued. A higher MV/GNP ratio, therefore a greater Market Value, makes this proof more difficult, and thus more trustworthy. In fact, this ratio is the highest estimate in Table 1, page 993 (S&P, 90 composite price index).
	

\section*{1.3}
According to the authors (and their growth theory valuation) stock market in August 1929 could be overvalued by 30 per cent (DeLong and Schleifer), if and only if, we have discarded the presence of intangible assets. However, these latter are economically significant to define the fair fundamental value of a stock, as Fisher (1930) stated. And thanks to new data that Fisher did not have back then, the writers have been able to proof the relevance of intangible capital. They reckon that intangible corporate assets such as investments in research, patents, monopoly rights are “The determining factor” The intangible capital was at least 0.57 times 1929 GNP, therefore the stock market was not overvalued, if ever undervalued. (BEA sources)
	

\section*{1.4)}
In order to encourage that firms would operate to the benefit of their owners (and not suffering from agency conflicts) the authors firstly assume that the rate of return on tangible corporate capital (after tax) is the same as the rate of return on intangible corporate capital and all other forms of it. Secondly, given equation 2’s usage of the tax rate, the authors also assume it constant, an unchanging tax policy being justified by their analysis of IRS and NIPA sources (Table 2). The rationale behind equation (2) is quite evident. To compute the after-tax NIPA profits, we just need to estimate the return on all corporate capital which is per assumption (1) the real interest rate (i) multiplied by tangible and intangible assets. After that we just need to account for taxes (τ_prof) and the trend growth rate of real output (g) 
	

\section*{1.5)}
The finding on the resource cost of intangible capital was shown to be considerable when compared to the stock of tangible capital. After arguing in point 1.3 that the only plausible situation in which the market is overvalued (DeLong and Shleifer) is one in which intangible assets are “null”. This equality KI = 0.61KT undermine the DeLong and Shleifer’s finding of overvaluation of the stock market at the end of August 1929. Due to the fact the only in case of close to zero estimate of intangible capital and remarkable high returns to tangible capital, the stock market would have been overvalued, as discussed in point 1.3.


	

\section*{1.6)}
Following equations (1) and (4), two assumptions need to be met to come up to the conclusion that the stock market in 1929 was overvalued. First, the amount of intangible capital must be extremely low and second the return on tangible assets must be truly high. The first assumption implies the second assumption, and as we know that tangible capital was high, intangible capital has to be low in order to have a low estimate of fundamental value. This means that the return on corporate capital must be remarkably high to create corporate profit shares as high as those seen in the 1920s. First of all, it is unlikely to have intangible assets to “null” value. Indeed, by using Fishers anecdotal evidence, research expenditures increased remarkably after World War 1. Innovation happened more frequently, not only via new patents, but also in the optimization of management processes. In the worst-case scenario, when intangibles have no value, the real after-tax return on tangible capital must be 5.9 per cent equation (4) which is too high with respect to the estimates based on national account data. As a matter of fact, these assumptions are implausible, given the observed data on corporate profits and national account data estimates of the real return on tangible capital.
	

\end{document}  